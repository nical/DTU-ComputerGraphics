\chapter{Week 11 - NURBS curves and surfaces}

\section{Part 1 - Curves}

The relation between degree order and number of knots is:
#knots = Degree + #ControlPoints + 1 = Degree + Order

\subsection{Bezier curves}

Figure \ref{beziergl} shows a bezier curve of order 4 displayed using opengl
commands (\textit{glMap1f} and \textit{glEvalCoord1f}).

We also implemented bezier curves of order n without opengl, using the difinition
of a bezier curve:
\begin{lstlisting}[caption=Generic bezier curve]
template<typename T>
T BezierCurve( T * points, int order, float t )
{
    int n = order-1;
    T result;
    for(int dim = 0; dim < T::Dimension; ++dim)
    {
        for(int i = 0; i <= n; ++i)
        {
            result[dim] += 
                  factorial(n) / (factorial(i)*factorial(n-i) )
                * pow(1-t,n-i)
                * pow(t,i)
                * points[i][dim];
        }
    }
    return result;
} 
\end{lstlisting}

Figure \ref{beziernogl} shows bezier curves of order 3, 4, 5, 6 and 7 using our implementation.

A bezier curve of order N requires N control points, (so the cubic bezier needs 4 control points). 

\image{Week11/bezier.png}{4th order (cubic) Bezier curve.}{0.5}{beziergl}
\image{Week11/bezier3_7.png}{Bezier curves of order 3 to 7.}{0.5}{beziernogl}

\subsection{Non uniform B-splines}

the glu library provides helper functions to render NURBS curves and surfaces. We
used it to render curves using the following knot vectors:
\begin{enumerate}
    \item \{ 0.0, 0.0, 0.0, 0.0, 0.10, 0.5, 0.90, 1.0, 1.0, 1.0, 1.0 \} 
    \item \{ 0.0, 0.0, 0.0, 0.0, 0.25, 0.5, 0.75, 1.0, 1.0, 1.0, 1.0 \} 
    \item \{ 0.0, 0.0, 0.0, 0.0, 0.40, 0.5, 0.60, 1.0, 1.0, 1.0, 1.0 \}
    \item \{ 0.0, 0.0, 0.0, 0.0, 0.50, 0.5, 0.50, 1.0, 1.0, 1.0, 1.0 \}
\end{enumerate}
The result is shown if figure \ref{nubs}

\image{Week11/nonUniformBSplines.png}{Non uniform B-splines of order 4 with different clamped knot vectors}{0.5}{ubs}

We used zeros and ones at the beginning and end of the knot vector in order to have
the curve start at the first control point and stop at the last. If we don't do so,
we obtain something similar to the uniform B-spline showed in figure \ref{ubs}.

Note that the 4th knot vector leads to the same curve as the cubic bezier.

\subsection{uniform B-splines}
The only difference between uniform and non-uniform B-splines lies in the knot vector.
Uniform B-splines introduce a new constraint which is that knots have to be equally
spaced. So we can render a uniform B-spline curve using the same method as with
non-uniform ones, with the following knot vector:
\{ 0.0, 0.1, 0.2, 0.3, 0.4, 0.5, 0.6, 0.7, 0.8, 0.9, 1.0 \}

Which results in the curve displayed in figure \ref{ubs}.

\image{Week11/uniformBSpline.png}{Uniform B-spline}{0.5}{ubs}

\section{Part 2 - Surfaces}
