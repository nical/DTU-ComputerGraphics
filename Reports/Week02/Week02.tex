\chapter{Week 2}

\section{Part 1}

The requested transformation can easily be achieved using the helper function
\syntax{gluLookAt} in modelview mode, as follows:

\begin{lstlisting}
glMatrixMode (GL_MODELVIEW);
glLoadIdentity ();
gluLookAt(2,2,2,0,0,0, 0,0,1);
\end{lstlisting}

\image{Week02/Part1.png}{Output image of Part 1.}{0.5}{img:w2p1}

\section{Part 2}
The same Transformation can be achieved by only composing with the following rotations:
\begin{lstlisting}[caption=Extract from Part2.cpp]
        glMatrixMode (GL_MODELVIEW);
        glLoadIdentity ();
        glRotatef(35, 1, 0, 0);
        glRotatef(-135, 0, 1, 0);
        glRotatef(-90, 1, 0, 0);
\end{lstlisting}
\image{Week02/Part2.png}{Output image of Part 2.}{0.5}{img:w2p2}

\section{Part 3}

%\image{Week02/Part3.png}{Output image of Part 3.}{0.5}{img:w2p3}

The general expression of a \textit{lookat} modelview matrix can be written as follows:

$$\begin{pmatrix}
 	X_x&		Y_x&           	Z_x&          	0\\
	 X_y&       Y_y&           	Z_y&          	0\\
	 X_z&       Y_z&           	Z_z&          	0\\
	-dot(X, eye)&  -dot(Y, eye)&  	-dot(Z, eye)&  	1\\
\end{pmatrix}$$
With X, Y and Z the following vectors:\\
Z = normalized(At - Eye)\\
X = normalized(cross(Up, zaxis))\\
Y = cross(Z, X)\\
Using dot() and cross() to denote the dot product and cross product of two vectors,
and normalized() the expression of the normalized vector (or unit vector) of a given vector.
