The objective of this part of the project was to design and to implement on
an \nxtOSEK{} system a digital controller for a vehicle in order to make it
follow a wall at a given distance.

\section{The steps needed to design a Digital Controller}

  The steps in designing and implementing a controller to follow a wall are
  very much the same as the one needed in implementing a digital controller
  for a motor:
  
  \begin{itemize}
    
    \item{} We have to compute new values for $\xi$ and $\omega_n$
      accordingly to our needs (i.e. how much time should it take to reach
      the required distance or how big should be the overshoot)
    \item{} Once we have computed them we have to use a graphical analysis
      method such as the Root Locus to design a Continuous time Laplace
      Transfer function to change the Closed Loop behaviour of our model
    \item{} Design a way to compute correctly our distance from the wall
    \item{} Implement the controller on the plataform

  \end{itemize}

  This time, however, other problems arose, such the fact that the
  controller for the motors was not designed correctly due to a bad
  motor identification or the fact that the ultrasonic distance sensors of
  the \nxt{} have limitations that hindered our design.
  
  Each one of these was taken in account during this last part of the
  project, as can be read in the next two sections.
