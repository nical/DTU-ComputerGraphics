%\documentclass[a4paper]{article}
\documentclass[9pt,a4paper]{scrreprt}

%\documentclass[10pt,a4paper,cleardoubleempty]{scrbook}
%\documentclass[9pt,a5paper,cleardoubleempty]{scrbook}

%\documentclass[]{scrreprt} % classe report di KOMA-Script
%\documentclass[h. . .i]{scrbook} % classe book di KOMA-Script
\usepackage{}
\usepackage[]{classicthesis-ldpkg}
\usepackage[parts,linedheaders,pdfspacing]{classicthesis}
\usepackage[english]{babel}
\usepackage[utf8]{inputenc}
\usepackage{indentfirst}
\usepackage{fullpage}

\hypersetup{pdfborder={0 0 0}}



\usepackage{braket}
\usepackage{amsmath}
\usepackage{graphicx}
%\usepackage{a4}

\newcommand{\Eqn}[1]{Equation~(\ref{#1})}
\newcommand{\Abs}[1]{\left|\:#1\:\right|}

\newcommand{\website}[2]{Sito web di \techname{#1}: \\ \url{#2}}
\newcommand{\paper}[3]{#1: \emph{#2} \\ \url{#3}}

\newcommand{\techname}[1]{{\sc #1}}
\newcommand{\cypher}[1]{{\tt #1}}
\newcommand{\syntax}[1]{{\tt #1}}
\newcommand{\codeconst}[1]{\lstinline{#1}}
\newcommand{\newterm}[1]{\emph{#1}}
\newcommand{\foreignword}[1]{\emph{#1}}

\newcommand{\isquarec}{\techname{I$^2$C}}
\newcommand{\lego}{\techname{LEGO}}
\newcommand{\nxt}{\lego{} \techname{MindStorms NXT}}
\newcommand{\nxtOSEK}{\techname{nxtOSEK}}
\newcommand{\BROFist}{\techname{BROFist}}
\newcommand{\SPAM}{\techname{SPAM}}
\newcommand{\PID}{\emph{Digital PID Controller}}
\newcommand{\RIS}{\techname{Robotics Invention System}}
\newcommand{\SciCos}{\techname{SciCos}}
\newcommand{\SciCosLab}{\techname{SciCosLab}}

\title{Introduction to Spacecraft Systems \& Design \\
       Assignment 01}
\author{Michele Bianchi}

\begin{document}

    \maketitle

    \section*{Question 2.1.1}

      Computing the \emph{exhaust velocity} required, as a first step, to compute
      the \emph{exhaust area} of the nozzle, $A_e$. To do that I've used
      the formula

      \begin{align*}
        F_0 &= F_{SL} + A_e p_{SL} \\
        F_0 - F_{SL} &= A_e p_{SL} \\
        A_e &= \frac{F_0 - F_{SL}}{p_{SL}}
      \end{align*}

      Which resulted in a $A_e$ equal to $2.54m^2$. To check if this result
      was correct I've checked with the actual data for the
      \techname{Vulcain} directly from Volvo Aereo specification sheet.
      The difference was around $3cm$ ($1.79m$ against the specification's
      $1.76m$), but this is due to various rounding, so I've used that value.

      The second data I needed was the \emph{exhaust pressure} $p_e$. The
      data we have about the engine contained the \emph{Combustion Chamber
      pressure}, so we have to first compute the \emph{nozzle throat
      pressure} $p_t$ and then, using the \emph{nozzle area ratio}, compute
      $p_e$. The formula used to compute $p_t$ is:

      \begin{align*}
        \frac{p_t}{p_c} &= {\left( \frac{2}{\gamma + 1} \right)}^{\frac{\gamma}{\gamma - 1}} \\
        p_t &= p_c {\left( \frac{2}{\gamma + 1} \right)}^{\frac{\gamma}{\gamma - 1}}
      \end{align*}

      Where $\gamma$ is the \emph{specific heat ratio} of the gas produced by the combustion, which, in our
      case, is $H_2O$, since the \techname{Vulcain} is a LOX + LH$_2$ engine. $\gamma$ is then, in our case,
      $1.33$. So, in the end, $p_t$ is:

      \begin{align*}
        p_t &= p_c {\left( \frac{2}{\gamma + 1} \right)}^{\frac{\gamma}{\gamma - 1}} \\
            &= 107 bar {\left( \frac{2}{1.33 + 1} \right)}^{\frac{1.33}{1.33 - 1}} \\
            &= 107 bar \cdot 0.858^{\frac{1.33}{0.33}} \\
            &= 107 bar \cdot 0.858^{4.03} \\
            &= 107 bar \cdot 0.5394 \\
            &= 57.719 bar
      \end{align*}

      To compute $p_e$ I divided it by the \emph{nozzle area ratio}, since we are assuming that the exhaust
      gasses obey the perfect gas law, then the value of PV has to be always the same. This means that, if we have
      a cross section that is 45 times larger than the starting position (assuming that we consider a volume that has an
      infinitesimal depth then the ratio between the two volumes is equal to the ratio of the two cross sections), then 
      the pressure has to be 45 times smaller, which leaves us with a $p_e$ of $1.28 bar$

      We can now proceed in computing the two values of $V_{e(vac)}$ and $V_{e(SL)}$.

      To compute the two values we can use equation 6.7:

      \begin{align*}
        F_{vac} &= \dot{m} V_{e(vac)} + A_e p_e \\
        F_{vac} - A_e p_e &= \dot{m} V_{e(vac)} \\
        \frac{F_{vac} - A_e p_e}{\dot{m}} &= V_{e(vac)} \\
        V_{e(vac)} &= \frac{1139000N - 2.54 m^2 \cdot 128000pa}{280 \frac{Kg}{s}} \\
                   &= 2906 \frac{m}{s}
      \end{align*}

      \begin{align*}
        F_{SL} &= \dot{m} V_{e(SL)} + A_e (p_e - p_{SL}) \\
        F_{SL} - A_e (p_e - p_{SL}) &= \dot{m} V_{e(SL)} \\
        \frac{F_{SL} - A_e (p_e - p_{SL})}{\dot{m}} &= V_{e(SL)} \\
        V_{e(SL)} &= \frac{885000N - 2.54 m^2 \cdot (128000pa - 100000pa)}{280 \frac{Kg}{s}} \\
                  &= 2915 \frac{m}{s}
      \end{align*}

    \section*{Question 2.1.2}

      To compute the \emph{effective exhaust velocity} ${V_e}^*$ for the
      \techname{STAR 37E} apogee kick motor we can use the formula 6.9 of
      the book:

      \begin{align*}
        {V_e}^* &= V_e + \frac{A_e (p_e - p_a)}{\dot{m}} \\
                &= I_{SP} \cdot g_0
      \end{align*}

      Where $I_{SP}$ is the specific impulse of the rocket engine and $g_0$
      is the gravity acceleration at the Earth surface.
      
      We know that $I_{SP}$ for a \techname{STAR 37E} motor is $280s$ as
      written in table 6.3, so we can easily compute the \emph{effective
      exhaust velocity} by sustituting the values in the formula:

      \begin{align*}
        {V_e}^* &= I_{SP} \cdot g_0 \\
                &= 280s \cdot 9.81 \frac{m}{s^2} \\
                &= 2746.8 \frac{m}{s}
      \end{align*}

    \section*{Question 2.1.3}
     
      The liquid propellant system appears to be the most efficient. As written on the book, in the case of solid propellants,
      we have a mean specific impulse in the range of 200 \textasciitilde{} 260 seconds (Even though the \techname{STAR 37E} has
      an $I_{SP}$ of 280 seconds), which can be improved by adding
      metallic particles into the solid propellant. The lowest specific impulse for liquid propellants, as stated on the book in
      table 6.1, is 276 seconds, which is higher than the maximum specific impulse for solid propellants. The systems used as examples
      on the book to explain the characteristics of liquid-propellant engines used both LOX + LH$_2$ as fuel, mixture that has a specific
      impulse of 390 seconds, which makes it almost double as efficient as the most solid propellants. This means that, for the same amount
      of propellant used, the thrust produced by a Liquid propellant is almost the double of a solid counterpart.

      This fact was also proven by the specific impulses of the two engines taken in account before. As stated in tables 6.2 and 6.3, the
      \techname{Vulcain} engine, that uses LOX + LH$_2$ liquid fuel, has a \emph{specific impulse in vacuum} of 432 seconds.
      The \techname{STAR 37E}, that uses a solid fuel, has a \emph{specific impulse in vacuum} of 280 seconds. These values reflect what we said
      previously.
    
    \section*{Question 2.1.4}

      
    \section*{Question 2.2.1}
      
      The payload that can be launched in orbit for that kind of configuration is roughly 8250Kg, based on figure 7.1.

      The different configuration between the two launcher configurations can be seen in table 7.5, where are listed the maximum payload
      capabilities for GTO insertions. The \techname{A44L} has a maximum capability of 4700Kg, while the \techname{A44LP} has a maximum
      capability of 4170Kg. This represents the maximum thrust we can get from the engines, so we can use the ratio between the two maximum
      payloads to scale the payload that can be launched to orbit \emph{2a}:

      \begin{align*}
        4700Kg : 4170Kg &= 8250Kg : x \\
        x &= \frac{4170Kg \cdot 8250Kg}{4700Kg} \\
        x &= 7320Kg
      \end{align*}

\end{document}
