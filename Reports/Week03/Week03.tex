\chapter{Week 3}

\section{Part 1}

A rasterization process like OpenGL uses cannot easily simulate real light, and
more importantly indirect lighting. In real life when sun rays enter a room, they
bounce on walls, which provides lighting for parts that dont directly face the light
sources. To emulate the effect of indirect lighting, OpenGL includes the ambient
term in its lighting equation. What it does is basically apply a base colour to
every vertice regardless of the orientaion of its normal. 
~\\
~\\
TODO: 2nd ambiant component ??

\image{Week03/Part1.png}{Output image of Part 1.}{0.5}{img:w2p1}

\section{Part 2}

\section{Part 3}
The distance between the eye and the object is not taken into account by the phong
equation - only the direction is - therefore, puting the eye at infinity does not
change the lighting of the object.\\
The vew vector - vector between the eye and the object, thus dependent on the
eye position - influences the specular component.\\
The object will not get illuminated - except for ambiant - by the light source if the later is placed at
infinity, because the equation includes an attenuation term accounting for the
distance between the objetc and the light source.\\
Binding the light source to the eye point results illumination only dependent on
the viewer, with the most illuminated parts being the one with normals directed
toward the viewer. For instance turning around a sphere would give the same
result whichever direction the viewer is looking the sphere from.\\
Directional lights correspond to point lights placed at infinity without attenuation
term. This kind of light are generally used to simulate the effect of powerful
and very distant light sources like the sun.\\

\section{Part 4}
